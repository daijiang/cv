\documentclass[11pt,oneside]{article}
% \documentclass[11pt,article,oneside]{memoir}   

% fonts
\usepackage{fontspec}
\setmainfont[Ligatures=TeX]{Gentium Basic}
% \defaultfontfeatures{
%     Path = /usr/local/texlive/2013/texmf-dist/fonts/opentype/public/fontawesome/ }
\usepackage{fontawesome}
% \usepackage[usenames,dvipsnames]{color}
\usepackage[usenames,dvipsnames]{xcolor}
\usepackage{xunicode}
% \usepackage{xltxtra}
% \defaultfontfeatures{Mapping=tex-text}
% \setromanfont[Ligatures={Common}, Numbers={OldStyle}, Variant=01]{Linux Libertine O}
% \setmonofont[Scale=0.8]{Monaco}

% links
\usepackage[pdfusetitle,pdftitle={Daijiang Li - vita},pdfauthor={Daijiang Li},
bookmarks=true,bookmarksnumbered=false,bookmarksopen=false,
breaklinks=false,pdfborder={0 0 0},backref=false,colorlinks=true,urlcolor=BlueViolet]
 {hyperref}

\usepackage{array}
\usepackage{datetime}
\usepackage{ragged2e}
% \usepackage{paralist}
\usepackage{fancyhdr,enumitem}

\usepackage{revnum}

\def\twidth{16cm}  % texwidth in cm. Letter paper size: 21.59 × 27.94 cm. 4.59/2 = 2.3 
\usepackage[letterpaper,text={\twidth,23.0cm},centering]{geometry}
\usepackage[compact,small,sf,bf]{titlesec}
% \usepackage{kpfonts,dsfont}

\RaggedRight
\sloppy

% Header and footer
% \usepackage{lastpage}
\pagestyle{fancy}
\lhead{\sf Curriculum Vitae: Daijiang Li}
\rhead{\sf\thepage} % of \pageref{LastPage}
\cfoot{}

% Date format
\newdateformat{dli}{\monthname~\THEYEAR}
\dli

% Head shade box.
\newcommand{\shadebox}[3][white]{\colorbox{#1}{\parbox{#2}{#3}}}

\def\maketitle{
\thispagestyle{plain}
\vspace*{-1.4cm}
\shadebox[gray!10]{\twidth}{\sf\color[rgb]{0.6,0,0} % bg color here
\hbox to \twidth{\begin{tabular}{p{6.0cm}}
\LARGE\textbf{Daijiang Li}\\[0.3cm]  % name here
\large\textbf{Curriculum Vitae}\\[0.6cm]
\normalsize{Last updated: \today}
\end{tabular}
\hfill\hbox{\fontsize{9}{12}\sf
\begin{tabular}{@{}rp{8.2cm}@{}}
									% Information here
\faHome & Department of Botany, University of Wisconsin-Madison, Madison, WI, 53705.\\
      \faPhone & +1 608-265-2191\\
      \faEnvelopeAlt & \texttt{\href{mailto:daijianglee@gmail.com}{daijianglee@gmail.com}}\\
      \faGlobe & \texttt{\href{www.daijiang.name}{www.daijiang.name}} \\
      \faTwitter & \texttt{\href{http://twitter.com/_djli}{@\_djli}}
\end{tabular}}}
}
\vspace*{0.2cm}}

% Section headings
\titlelabel{}
\titlespacing{\section}{0pt}{2ex}{1ex}
\titleformat*{\section}{\color[rgb]{0.6,0,0}\large\sf\bfseries}
\titlespacing{\subsection}{0pt}{1ex}{0.5ex}

% Miscellaneous dimensions
\setlength{\parskip}{0ex}
\setlength{\parindent}{0em}
\setlength{\headheight}{15pt}
\setlength{\tabcolsep}{0.15cm}
\clubpenalty = 10000
\widowpenalty = 10000
\setlist{nolistsep}
% \setlist{itemsep=0pt}
\setdescription{labelwidth=1.2cm,leftmargin=1.5cm,labelindent=1.5cm,font=\rm}

%% Custom hanging indent for vita items
\def\ind{\vspace{0.3em} \hangindent=1.25 true cm\hangafter=1 \noindent }
%\def\ind{\hangindent=18pt\hangafter=1 \noindent}
%% No bullets on labels
% \def\labelitemi{~}
% \renewcommand{\labelitemii}{~}

%%%%%%%%%%%%%%%%%%%%%%%%%%%%%%%%%%%%%%%%%%%%%%%%%%%%%%%%
\begin{document}
\maketitle

\section{Education}
\ind 2011 -- Now, Ph.D. student, Department of Botany, University of Wisconsin-Madison\\
	\ind \hspace{1.25cm} \footnotesize Committee: Donald M. Waller (chair), Ellen I. Damschen, Thomas J. Givnish, Anthony R. Ives, Bret R. Larget\\ \vspace{-0.3em}
	\ind \hspace{1.25cm} Dissertation: Long-term dynamics of pine barrens in central Wisconsin
\normalsize

\ind 2013 -- Now, M.S. Biometry, Department of Statistics, University of Wisconsin-Madison

\ind 2008 -- 2011, M.S., School of Life Science, Sun Yat-sen University, China

\ind 2004 -- 2008, B.S., School of Life Science, Yunnan University, China


\section{Research Interests}
\ind Plant community assembly (taxonomic, functional and phylogenetic), Diversity and ecosystem functioning, Network analysis, Species co-occurrence, Multi-level model, Global change.

\section{Academic Appointments}
\ind 2014 -- 2015, Graduate Student Research Assistant, Department of Botany, University of Wisconsin-Madison.

\ind 2012 -- 2013, Graduate Student Teaching Assistant, Department of Botany, University of Wisconsin-Madison.

\ind 2011 -- 2012, Graduate Student Research Assistant, Department of Botany, University of Wisconsin-Madison.

\ind 2008 -- 2011, Graduate Student Research Assistant, School of Life Science, State Key Laboratory of Biocontrol, Sun Yat-sen University, China.

\section{Publications}
\noindent\emph{Journal articles}
\vspace{0.05in}
\begin{revnumerate}
\itemsep -0.1em 
	\item \textbf{Daijiang Li} and Donald Waller. \emph{In press}. \href{http://www.esajournals.org/doi/abs/10.1890/14-0893.1}{Drivers of observed biotic homogenization in pine barrens of central Wisconsin.} \emph{Ecology}.

	\item \textbf{Daijiang Li}, Shaolin Peng, Baoming Chen. (2013). \href{http://www.daijiang.name/pdf/D_Li_2012_plantsoil.pdf}{The effects of leaf litter evenness on decomposition depend on which plant functional group is dominant.} \emph{Plant and Soil}. 365:1-2, 255-266. 

	\item Baoming Chen, Shaolin Peng, Carla M. D'Antonio, \textbf{Daijiang Li}, Wentao Ren. (2013). \href{http://www.plosone.org/article/info%3Adoi%2F10.1371%2Fjournal.pone.0066289}{Non-Addiitive Effects on Decomposition from Mixing Litter of the Invasive \emph{Mikania micrantha} H.B.K. with Native Plants.} \emph{PLoS ONE}. 8(6): e66289.

	% \vspace{-0.3em}
	% \ind \hspace{0cm} Some noteable thing about this paper...

	\item Bosun Wang, Shaolin Peng, \textbf{Daijiang Li}, Ting Zhou. (2009). Research progress on \emph{Merremia boisiana}. \emph{Chinese Journal of Ecology}. 28 (11), 2360-2365.
\end{revnumerate}

\noindent\emph{Manuscripts in review or preparation}
\vspace{0.05in}
\begin{revnumerate}
\itemsep -0.1em 
	\item \textbf{Daijiang Li} and Donald Waller. \emph{In press}. Long-term shifts in the forces driving community assembly inferred from plant co-occurrence patterns. \emph{Submitted}.
\end{revnumerate}


\section{Awards and Grants}
\ind 2013 -- 2014, The Flora Aeterna Research Grant, \$1,500. Department of Botany, University of Wisconsin-Madison

\ind 2012 -- 2013, The Flora Aeterna Research Grant, \$5,000. Department of Botany, University of Wisconsin-Madison

\ind 2012 -- 2013, Davis Research Grant, \$1,500. Department of Botany, University of Wisconsin-Madison 

\ind 2013, Phi Kappa Phi Honor Society (declined), University of Wisconsin-Madison

\ind 2009, 2010, Education Scholarship for Graduate Student (Fellowship), Sun Yat-sen University, China

\ind 2009, Excellent prize of presentation in English, Sun Yat-sen University, China

\ind 2008, Second Award of National Scholarship, Yunnan University, China

\ind 2006, 2007, Second Award of Donglu Scholarship, Yunnan University, China



\section{Presentations and Posters}
\ind \textbf{Daijiang Li}, Donald Waller, 2014 August, Increases in native, not exotic, species plus succession promote biotic homogenization in the sand plain plant communities of central Wisconsin. \emph{99th Ecological Society of America Annual Meeting}, Sacramento, California, USA. (Oral presentation).

\ind \textbf{Daijiang Li}, Donald Waller, 2014 March, Long-term dynamics in pine barrens of central Wisconsin. \emph{Wisconsin Ecology Annual Spring Symposium}, Madison, Wisconsin. (Oral presentation).


% \section{Professional Experience}


\section{Teaching}
\ind 2014 August 25-26, Instructor, Software Carpentry Bootcamp, UW-Madison.

\ind 2013 Fall, Teaching Assistant, Vegetation of Wisconsin (Bot 455).

\ind 2013 Spring, Teaching Assistant, Introductory Biology (Bio 152).

\ind 2012 Fall, Teaching Assistant, Vegetation of Wisconsin (Bot 455).

\ind 2010 Fall, Teaching Assistant, Restoration Ecology, Sun Yat-sen University, China.



\section{Undergraduate Mentoring / Research Training}
\ind 2014, Kelly Wallin, Prepare leaf samples for leaf nutrient analysis.

\ind 2014, David Barfknecht, Collect and measure plant functional traits. 

\ind 2014, Madeline Grupper, Prepare leaf samples for leaf nutrient analysis.

\ind 2013, Alex Arena, Collect and measure plant functional traits.

\ind 2013, Anna Francke, Changes in overstory structure of pine barrens in Wisconsin from 1958 to 2012.


\section{Professional Service and Outreach}
\ind Manuscript reviewer for: \emph{Journal of Ecology}.

\ind 2014, Darwin day prensenter, UW-Madison.

\ind 2013 -- 2014. Student representative to the Biology Colloquium Committee, Botany department, UW-Madison.

\ind 2013 -- 2014. Graduate student Committee, Botany department, UW-Madison.

\ind 2013 Oct. 6-11, volunteer of the fifth world conference on ecological restoration.

\ind 2012 -- 2013. Department Technical Committee, Botany department, UW-Madison.


\section{Skills}
\ind Plant identification, R, \LaTeX, Linux, Git.

\section{References}
\ind Dr. Donald. M. Waller (PhD Advisor) \\
John T. Curtis Professor of Department of Botany\\
University of Wisconsin-Madison, Madison, WI 53706\\
608-262-2191, \href{mailto:dmwaller@wisc.edu}{dmwaller@wisc.edu}
\bigskip

\ind Dr. Anthony R. Ives (Committee member) \\
Professor of Department of Zoology\\
University of Wisconsin-Madison, Madison, WI 53706\\
608-262-9226, \href{mailto:arives@wisc.edu}{arives@wisc.edu}
\bigskip

\ind Dr. Ellen I. Damschen (Committee member) \\
Professor of Department of Zoology\\
University of Wisconsin-Madison, Madison, WI 53706\\
608-262-4437, \href{mailto:damschen@wisc.edu}{damschen@wisc.edu}

\end{document}